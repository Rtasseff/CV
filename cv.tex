\documentclass[margin,line]{res}


\oddsidemargin -.5in
\evensidemargin -.5in
\textwidth=6.0in
\itemsep=0in
\parsep=0in

\newenvironment{list1}{
  \begin{list}{\ding{113}}{%
      \setlength{\itemsep}{0in}
      \setlength{\parsep}{0in} \setlength{\parskip}{0in}
      \setlength{\topsep}{0in} \setlength{\partopsep}{0in} 
      \setlength{\leftmargin}{0.17in}}}{\end{list}}
\newenvironment{list2}{
  \begin{list}{$\bullet$}{%
      \setlength{\itemsep}{0in}
      \setlength{\parsep}{0in} \setlength{\parskip}{0in}
      \setlength{\topsep}{0in} \setlength{\partopsep}{0in} 
      \setlength{\leftmargin}{0.2in}}}{\end{list}}


\begin{document}

\name{Ryan A. Tasseff \vspace*{.1in}}

\begin{resume}
\section{\sc Contact Information}
\vspace{.05in}
\begin{tabular}{@{}p{3in}p{4in}}
Institute for Systems Biology & {\it Voice:}  (+33 6) 46 44 10 89  \\            
401 Terry Avenue North   & {\it E-mail:}  rtasseff@systemsbiology.org   \\         
Seattle, WA 98109-5234 & {\it WWW:} systemsbiology.org\\         
\end{tabular}

%
%\section{\sc Objective}
%Seeking research position in computational cell biology that will benefit from my doctorate experience in both computational and experimental studies; 
%strong interest in identification of techniques and strategies for the analysis of complex cellular 
%systems, with a focus on problems relevant to human health and close collaboration with experimental biologists.    
%\section{\sc Summary}
%Computational biology, process modeling and nonlinear dynamics of complex systems. 
%Emphasis on the integration of knowledge and data with appropriate mathematical frameworks. 
%Strong interest in working with/for industry and commercial research organizations.


\section{\sc Education}
{\bf Cornell University}, Ithaca, New York USA\\
%{\em Department of Statistics} 
\vspace*{-.1in}
\begin{list1}
\item[] Ph.D. Chemical and Biomolecular Engineering, January 2012 
\begin{list2}
\vspace*{.05in}
\item Dissertation Topic:  ``Reconstruction and Analysis of the Molecular 
Programs Involved in Deciding Mammalian Cell Fate'' 
%\item Dissertation Topic:  ``Hierarchical Models for Multiple Ratings
%  in Performance-Based\\ \hspace*{1.23in} Student Assessments.'' 
\item Concentrations: Applied Mathematics and Computational Methods, Cellular and Molecular Medicine, Classical and Statistical Thermodynamics 
\item Advisor/ Committee Chair:  Jeffrey D. Varner
\end{list2}
\end{list1}

{\bf University of Florida}, Gainesville, Florida USA\\
%{\em Department of Mathematics and Statistics} 
\vspace*{-.1in}
\begin{list1}
\item[] B.S., Chemical Engineering,  June, 2006
\begin{list2}
\vspace*{.05in}
\item Minor in Mathematics and Advanced Option in Physical Chemistry
\end{list2}
\end{list1}


\section{\sc Honors and Awards}
Edna and William C. Hooey Fellowship, Cornell University 2011

\vspace*{-2.5mm}
National Science Foundation IGERT Fellowship in Nonlinear Dynamics, Cornell University 2008

\vspace*{-2.5mm}
Graduated Magna Cum Laude, Honors in Chemical Engineering, University of Florida, 2006

\vspace*{-2.5mm}
University Scholars Research Grant and Best Undergraduate Paper, University of Florida, 2005

\vspace*{-2.5mm}
Florida Bright Futures Award for 100\% paid tuition, 2001



\section{\sc Academic Experience}
{\bf Institute for Systems Biology}, Seattle, WA USA \hfill {\bf 2011 - present}

{\em Research Scientist}\\
Participating in clinical-genomic research  
for ISB-INOVA health care system collaboration.
Primary goal: identify statistical clinical-genomic association 
and develop general analysis pipelines. 
Assisted with statistical analysis and overall analysis design for pilot 
project on pre-term birth involving 2k+ whole genome sequences.
Contributed to drafting first joint ISB-INOVA research paper.
Participated in hiring committee for expanding project team. 

Visiting scientist at Institut Génétique Biologie Moléculaire Cellulaire
in llkirch-Graffenstaden, France. 
Primary goal: Work with yeast biologist to apply machine learning techniques for the identification of 
key cell cycle features within large, heterogeneous datasets and 
developing spatial temporal models of yeast dynamics within a microfludic chip. 

{\em Postdoctoral research on computational systems biology}\\
Participated in ISB-Procter and Gamble (P\&G) collaboration.  
Primary goal: integration of systems biology methods with current research pipelines at P\&G
related to Epithelial biology.
Worked directly with P\&G biologist to formulate measures 
for systematic identification of biomarkers and therapeutic targets
given existing high-throughput data.
Assisted in experimental design for improving identification of key features.
Designed novel mathematical framework for dynamic analysis of periodic behavior in mouse hair-cycle.
Modeled and simulated skin barrier formation.
Multiple targets for inflammatory skin diseases and hair growth 
modulation submitted internally to P\&G for potential patent submission.



{\bf Cornell University}, Ithaca, New York USA \hfill {\bf 2006 - 2011}

{\em Ph.D.~research on computational and experimental molecular cell biology.}\\   
%Constructed mechanistic model of crosstalk between growth factor and androgen signaling in
%prostate cancer cell line.  We identified both androgen independent and dependent states under 
%a single model framework and provided mechanistic insight on an experimentally observed synergy.
Constructed and analyzed mechanistic mathematical models of biomolecular interaction networks.
Networks associated with progression of androgen independent prostate cancer 
and differentiation of hematopoietic GM precursor cells. 
Identified robust yet fragile subnetworks relating to the cellular infrastructure.
Employed both experimental and computational techniques to characterize a feedback control circuit
in programed differentiation.
%
%Constructed detailed mechanistic model of Retinoic Acid induced differentiation in leukemia cell line.
%We introduced a means to systematically rank experimental targets by uniqueness and
%identified a robust yet fragile subnetwork relating to the cellular infrastructure.
%
%Employed both experimental and computational techniques to characterize a feedback control circuit
%in programed differentiation.
%We experimentally correlated activity and inhibition of a critical subsystem to molecular and phenotypic events during 
%Retinoic Acid induced differentiation and 
%computationally uncovered a bifurcation in the system consistent an apparent cellular memory effect. 
% 
\begin{list2}
\item Completed both computational and experimental research in cell biology
\item Advisor and team leader for 30+ undergraduate/Masters/PhD research projects
\item Designed, organized and maintained the departments first mammalian cell culture facility
\end{list2}

%{\em Ph.D.~and Masters level coursework.}\\ 
%Selected courses: Statistical Physics (PHYS653),Nonlinear Dynamics and Chaos (PHY578), 
%The Nucleus (BIOBM639) and Systems Biology in Medicine (CHEME544).
%
%{\em Graduate student advisor and team leader for 30+ undergraduate/Masters/PhD research projects.}\\
%Mentored 30+ students working on multiple projects in the Varner lab.   
%Although, students were primarily undergraduates I was afforded the opportunity to mentor a rotational PhD student, 
%three masters students and an extremely motivated high school student.

%{\em Designed, organized and maintained Olin Hall shared mammalian cell culture facility.}\\
%I oversaw purchasing of lab supplies and equipment, with \$40K+ budget,
%for construction of a new Olin Hall lab space.  
%The facility was the first space available for independent experimental research in my particular group.

{\em Instructed Course for Masters of Engineering program.}
%I was responsible for planning curriculum for a new class being offered to incoming masters student.
%The focus was to train students in general mammalian culture and molecular biology techniques. 
%Personal responsibilities included maintenance of the facility, design of the experiments, planning of the curriculum
%and general instruction.  
\begin{list2}
\item Novel Masters Program: Medical and Industrial Biotechnology
\item CHEME 5490, MIB Molecular Biology Lab
\item Designed the curriculum for Mammalian Cell Culture  
\end{list2}

{\em Teaching Assistant}
%Duties at various times have included 
%office hours and occasional lecturing, grading and advising.
\begin{list2}
\item CHEME 2880, Biomolecular Engineering: Fundamentals and Applications\\
\end{list2}


\section{\sc Industry Experience}
{\bf Cornell / General Electric}, Albany, New York USA \hfill {\bf 2008 and 2009}

\vspace{-.35cm}
{\em Cornell Business of Science and Technology Initiative member}\\ 
Assisted in analysis of a full supply chain model
for estimating profits from ventures into woody biomass gasification.      
\vspace{-.2cm}

{\bf University of Florida / Progress Energy}, Crystal River, Florida USA \hfill {\bf 2005 - 2006}

\vspace{-.35cm}
{\em Integrated Product and Process Design Member}\\ 
Lead development of software for real-time predictions
of facility cooling processes to maintain environmentally safe operations.
\vspace{-.2cm}

{\bf Dow Chemical Company},  USA \hfill {\bf 2003 - 2004}

\vspace{-.35cm}
{\em Rotating Cooperative}\\
Computational design of distillation columns and separations processes in Freeport, TX.\\
Research and development of pharmaceutical soluble polymers in Plaquemine, LA.\\
Process control and optimization in Pittsburg, CA.  


\section{\sc Skills} 
\begin{list2}
\item Programming languages: proficient in Python, basic skills in C++, Java, Fortran and Unix shell scripts 
\item Mathematical platforms:  Matlab, Octave, R and Aspen
\item Operating Systems:  Mac OSX, Linux, Windows.
\item Experience with cloud/commodity computing (Amazon EC2)
\item Mammalian cell culture techniques
\item Biochemical techniques: Fluorescence Activated Cell Sorting, cytometry, cloning, immunochem.
\item Familiarity with high-throughput data (mRNA microarray, whole genome sequencing)
\end{list2}



\section{\sc Publications}
{\bf Ryan Tasseff}, Anjali Bheda-Malge, Teresa DiColandrea, Charles C. Bascom, Robert J. Isfort and Richard Gelinas.
Mouse hair cycle expression dynamics modeled as coupled mesenchymal and epithelial oscillators.  
{\it PLoS Comp. Bio. } Accepted in Sep. 2014. 

Holly A. Jensen, Lauren E. Styskal, {\bf Ryan Tasseff}, et al. 2013
The Src-Family Kinase Inhibitor PP2 Rescues Inducible Differentiation Events 
in Emergent Retinoic Acid-Resistant Myeloblastic Leukemia Cells.
{\it PLoS ONE} 8(3): e58621. doi:10.1371/journal.pone.0058621

{\bf Ryan Tasseff}, Satyaprakash Nayak, Sang Ok Song, Andrew Yen and Jeffrey Varner. 2011. 
Modeling and analysis of retinoic acid induced differentiation of uncommitted precursor cells.
{\it Integrative Biology, DOI: 10.1039/c0ib00141d.}. 

Young-Min Ban, {\bf Ryan A Tasseff}, and Dmitry I. Kopelevich. 2011.
Non-adiabatic Dynamics of Interfacial Systems: A Case Study of a Nanoparticle Penetration into a Lipid Bilayer.
{\it Molecular Simulation}, 37(7) 525-536. 

{\bf Ryan A. Tasseff}, and Jeffrey D. Varner. 2011.
Mathematical Models in Biotechnology.
{\it Comprehensive Biotechnology}, 2nd edition ISBN: 9780444533524

Timon H. Stasko, Robert J. Conrado, Andreas Wankerl, Rodrigo Labatut, {\bf Ryan Tasseff}, et al. 2010. 
Mapping Woody-Biomass Supply Costs Using Forest Inventory and Competing Industry Data. 
{\it Biomass and Bioenergy}.  doi:10.1016/j.biombioe.2010.08.044

Reiterer G, Chen L, {\bf Tasseff R}, Varner JD, Chen CY and Yen A. 2010.
Raf associates with phosphorylated nuclear BubR1 during endoreduplication 
induced by JAK inhibition.  
{\it Cell Cycle} 9(16):3297-304

{\bf Tasseff R}, Nayak S, Salim S, et al. 2010. 
Analysis of the Molecular Networks in Androgen Dependent and 
Independent Prostate Cancer Revealed Fragile and Robust Subsystems. 
{\it PLoS ONE} 5(1): e8864. doi:10.1371/journal.pone.0008864

Dmitry I. Kopelevich, Jean-Claude Bonzongo, {\bf Ryan A. Tasseff}, et al.
2008. Potential Toxicity of Fullerenes and Molecular Modeling of Their 
Transport across Lipid Membranes. 
{\it Nanoscience and Nanotechnology}:233-260. 
Copy Right John Wiley \& Sons, Inc.

{\bf Ryan Tasseff}, Dr. Dmitry Kopelevich. 2006. 
Molecular Modeling of Nanoparticle Transport 
across Lipid Bilayers.
{\it University of Florida: Journal of Undergraduate Research} 7(4)



%\section{\sc Papers Submitted}
%{\bf Ryan A Tasseff}, Anirikh Chakrabarti, Linlin Yang, Robert Dromms, Anita Gokhlay, 
%Anastasiya Dubrovina, Megan Altmire, Kerianne Dobosz, Woojin Kim and Jeffrey D. Varner.
%UNIVERSAL: An Extensible Physiochemical Model Generation and Analysis Framework for Mac OS X.
%{\it Bioinformatics}.


\section{\sc Papers in preparation}
INOVA-ISB Research Network.
A clinical and whole genome sequencing study of preterm birth.

{\bf Ryan Tasseff}, Johanna Congleton†, Andrew Yen† and Jeffrey D. Varner.
Investigation of the cRaf interactome and steady-state multiplicity in 
Retinoic Acid-Induced Differentiation of HL-60 cells.

\section{\sc Conference Presentations}
ECMTB 9th European Conference on Mathematical and Theoretical Biology 2014, Gothenburg, Sweden (oral) \\
Biocellion: Accelerating multicellular biological simulation.
{\bf Ryan Tasseff}, Seunghwa Kang, Simon Kahan, Ilya Shmulevich and Nick Flann.

SBE International Conference on Biomolecular Engineering 2011, San Fransisco, CA (poster)\\
Modeling and Analysis of the Retinoic Acid Induced Proliferation and Differentiation Program of HL-60.
{\bf Ryan Tasseff}, Satyaprakash Nayak, Sang Ok Song, Andrew Yen and Jeffrey D. Varner.

AICHE Annual Meeting 2010, Salt Lake City, UT - In Silico Biology (oral) \\
Modeling and Analysis of the Retinoic Acid Induced Proliferation and Differentiation Program of HL-60.
{\bf Ryan Tasseff}, Satyaprakash Nayak, Sang Ok Song, Andrew Yen and Jeffrey D. Varner.

ACS National Meeting 2010, San Francisco, CA - Biotechnology (poster)\\
Modeling and Analysis of the Retinoic Acid Induced Proliferation and Differentiation Program of HL-60.
{\bf Ryan Tasseff}, Satyaprakash Nayak, Sang Ok Song, Andrew Yen and Jeffrey D. Varner.

AICHE Annual Meeting 2008, Philadelphia, PA - Systems Biotechnology II (oral)\\
Mathematical Modeling and Analysis of the Role of the BLR1 Protein and MAPK Activation 
in the Growth-Arrest and Differentiation Program of a Model Adult Stem-Cell.
Jeffrey D. Varner, {\bf Ryan Tasseff}, Satyaprakash Nayak and Andrew Yen.

AICHE Annual Meeting 2008, Philadelphia, PA - Engineering Life Sciences (poster)\\
Formulation and Analysis of An Ultrascale Protein Interaction Network Involved In the Androgen Response of 
Prostate Cancer Epithelial Cells.
{\bf Ryan A. Tasseff}, Satyaprakash Nayak, Poorvi Kaushia, Noreen Rizvi, Saniya Salim, Jeffrey D. Varner.

AICHE Annual Meeting 2007, Salt Lake City, UT - Bioengineering (poster)\\
Identification of Fragile Mechanisms in the Human Complement Cascade Is Sensitive 
to the Choice of Numerical Method for the Solution of the Sensitivity Equations.
{\bf Ryan A. Tasseff}, Jeffrey D. Varner, Satyaprakash Nayak, Thomas J. Mansell, Deyan Luan.

AICHE Annual Meeting 2005, Cincinnati, OH - Transport in Nanoscale (oral)\\
Modeling of Transport of Nanoparticles across a Lipid Bilayer.
{\bf Ryan A. Tasseff} and Dmitry I. Kopelevich.


\section{\sc Other}
Peer Reviewer for Oxford Journal, {\it Bioinformatics}

\end{resume}
\end{document}




